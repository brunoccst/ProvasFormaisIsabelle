\documentclass{article}
\usepackage[utf8]{inputenc}

\title{Especificação de algoritmos com Isabelle}
\author {Bruno Carvalho, Guilherme Schmitt, Henri Dias,
\\João Fanti, Matheus Santos}
\date{Junho de 2016}

\usepackage{natbib}
\usepackage{graphicx}
\usepackage{amssymb}

\begin{document}

\maketitle

\section{Primeiro problema}
\subsection{Especificação}
\textbf{mult}: N x N $\rightarrow$ N
\\requer~T
\\garante mult(x,y)= x * y
\\mult(x,y) = multacc(x,y,0)
\\\textit{onde}
\\
\textbf{multacc}: N x N x N $\rightarrow$ N
\\requer T
\\garante multacc(x,y,z) = x * y + z
\\invariante $\forall$k $\in$ N.  multacc(m0,n0,a0) = mk * nk + ak
\\\textit{multacc01}: multacc(0,n,a) = a
\\\textit{multacc02}: multacc(k+1,n,a) = multacc(k,n,a + n)

\subsection{Exemplo de execução}
mult(2, 3)
\\= multacc(2, 3, 0)
\\= multacc(1, 3, (0+3))
\\= multacc(0, 3, (3+3))
\\= 6

\subsection{Teorema th01: multacc(x,y,z) = x * y + z}
P $\triangleq$ $\forall$ x,y,z $\in$ N. multacc(x,y,z) = x * y + z
\\P(x) $\triangleq$ $\forall$ y,z $\in$ N. multacc(x,y,z) = x * y + z
\\
\\\subsubsection{Base:} P(0) $\triangleq$ $\forall$ y,z $\in$ N. multacc(0,y,z) = 0 * y + z
\\
\\Seja y0 uma variável arbitrária 
\\Seja a0 uma variável arbitrária 
\\Mostrar multacc(0,y0,a0) = 0 * y0 + a0
\\
\\\textit{Prova:}
\\multacc(0,y0,a0) (por definição)
\\0 * y0 + a0 (por simp)
\\a0 (por multacc01)
\\(0 * y0) + a0 (por aritmética)
\\0 * y0 + a0 (por simplificação)
\\
\\\subsubsection{Indução:}
Seja x0 uma variável arbitrária
\\Assumir como hipótese P(x0) $\triangleq$ $\forall$ y,z $\in$ N. multacc(x0,y,z) = x0 * y + z
\\Demonstrar P(x0 + 1) $\triangleq$ $\forall$y,z $\in$ N. multacc(x0 + 1,y,z) = (x0 + 1) * y + z
\\
\\Seja y0 uma variável arbitrária 
\\Seja a0 uma variável arbitrária 
\\Mostrar multacc(x0 + 1,y0,a0) = (x0 + 1) * y0 + a0
\\
\\\textit{Prova:}
\\multacc x0 y0 (a0 + y0) (por multacc02)
\\x0 * y0 + (a0 + y0) (por hipótese) 
\\x0 * y0 + a0 + y0 (por aritmética)
\\(x0 * y0) + (y0 * 1) + a0 (por álgebra)
\\(y0 * x0) + (y0 * 1) + a0 (por álgebra)
\\y0 * (x0 + 1) + a0 (por álgebra)
\\(x0 + 1) * y0 + a0 (por álgebra)

\subsection{Teorema th02: mult(x,y) = x * y}
P $\triangleq$ $\forall$x,y $\in$ N. mult(x,y) = x * y
\\P(x) $\triangleq$ $\forall$y $\in$ N. mult(x,y) = x * y
\\
\\\subsubsection{Base:} P(0) $\triangleq$ $\forall$y $\in$ N. mult(0,y) = 0 * y
\\
\\Seja y0 uma variável arbitrária
\\Mostrar mult(0,y0) = 0 * y0
\\
\\\textit{Prova:}
\\multacc 0 y0 0 (por mult01)
\\0 (por multacc01)
\\0 * y0 (por aritmética)
\\
\\\subsubsection{Indução:}
Seja x0 uma variável arbitrária
\\Assumir como hipótese P(x0) $\triangleq$ $\forall$y $\in$ N. mult(x0,y) = x0 * y
\\Demonstrar P(x0 + 1) $\triangleq$ $\forall$y $\in$ N. mult(x0 + 1,y) = (x0 + 1) * y
\\
\\Seja y0 uma variável arbitrária
\\Mostrar mult(x0 + 1,y0) = (x0 + 1) * y0
\\
\\\textit{Prova:}
\\mult (Suc x0) y0 (por simplificação)
\\multacc (Suc x0) y0 0 (por mult01)
\\multacc x0 y0 (0 + y0) (por multacc02)
\\multacc x0 y0 y0 (por simplificação)
\\x0 * y0 + y0 (por th01)
\\y0 * x0 + y0 (por algebra)
\\(y0 * x0) + (y0 * 1) (por algebra)
\\y0 * (x0 + 1) (por algebra)
\\(x0 + 1) * y0 (por algebra)
\\(Suc x0) * y0 (por simplificação)     

\section{Segundo problema}

\subsection{Fatorial}

\subsubsection{Especificação}
\textbf{fat}: N x N $\rightarrow$ N
\\requer x $\ge$ 0
\\garante fat(x) = x!
\\fat(x) = fataux(x,1)
\\onde
\\fataux: N x N $\rightarrow$ N
\\requer x $\ge$ 0
\\garante fataux(x,y) = x! * y
\\invariante $\forall$k $\in$ N. fataux(m0,n0) = mk! * nk
\\\textit{fataux01}: fataux 0 a = a
\\\textit{fataux02}: fataux(n,a) = fataux((n - 1), (a * n)), se n > 0

\subsubsection{Exemplo de execução}
fat(3)
\\= fataux (3, 1)
\\= fataux(2, (1 * 3))
\\= fataux(1, ((1 * 3) * 2))
\\= fataux(0, (((1 * 3) * 2) * 1))
\\= (((1 * 3) * 2) * 1)
\\= ((3 * 2) * 1)
\\= (6 * 1)
\\= 6

\subsubsection{Teorema th01: $\forall$cont. fataux n cont = n! * cont}
P $\triangleq$ $\forall$x,y $\in$ N. fataux(x,y) = x! * y
\\P(x) $\triangleq$ $\forall$y $\in$ N. fataux(x,y) = x! * y
\\
\\\textbf{Base}: P(0) $\triangleq$ $\forall$ y $\in$ N. fataux(0,y) = 0! * y
\\
\\Seja cont0 uma variável arbitrária 
\\Mostrar fataux(0, cont0) = 0! * cont0
\\
\\\textit{Prova:}
\\cont0 (por fataux01)
\\1 * cont0 (por aritmética)
\\0! * cont0 (por fatorial)
\\
\\\textbf{Indução}:
\\Seja m0 uma variável arbitrária
\\Assumir como hipótese P(m0) $\triangleq$ $\forall$y $\in$ N. fataux(m0,y) = m0! * y
\\Demonstrar P(m0 + 1) $\triangleq$ $\forall$y $\in$ N. fataux(x0 + 1,y) = (x0 + 1)! * y
\\
\\Seja a0 uma variável arbitrária 
\\Mostrar fataux(m0 + 1, a0) = (m0 + 1)! * a0
\\
\\\textit{Prova:}
\\fataux (Suc m0) a0 (por simplificação)
\\fataux m0 ((Suc m0) * a0) (por fataux02) 
\\m0! * ((Suc m0) * a0) (por hipótese)
\\m0! * (Suc m0) * a0 (por aritmética)
\\(Suc m0)! * a0 (por simplificação)
\\(m0 + 1)! * a0 (por simplificação)

\subsubsection{Teorema th02: fat n = n!}
P $\triangleq$ $\forall$x $\in$ N. fat(x) = x!
\\P(x) $\triangleq$ fat(x) = x!
\\
\\\textbf{Base}: P(0) $\triangleq$ fat(0) = 0!
\\Mostrar fat(0) = 0!
\\
\\\textit{Prova:}
\\1 (por fat01)
\\0! (por th01)
\\
\\\textbf{Indução}:
\\Seja m0 uma variável arbitrária
\\Assumir como hipótese P(m0) $\triangleq$ fat(m0) = m0!
\\Demonstrar P(m0 + 1) $\triangleq$ fat(m0 + 1) = (m0 + 1)!
\\
\\Mostrar fat(m0 + 1) = (m0 + 1)!
\\
\\\textit{Prova:}
\\fat (Suc m0) (por simplificação)
\\fataux (Suc m0) 1 (por fat02)
\\(Suc m0)! * 1 (por th01)
\\(Suc m0)! (por aritmética)
\\(m0 + 1)! (por simplificação)

\subsection{Somatório}
\subsubsection{Especificação}
\textbf{somatorio}: N $\rightarrow$ N 
\\requer: x $\ge$ 0
\\garante somatorio(n) = $$\sum_{i=1}^{n} i^{2} = (n/6) * (n + 1) * (2 * n + 1)$$ 
\\somatorio(n) = sumaux(n,0)
\\onde
\\\textbf{sumaux}: N x N $\rightarrow$ N
\\requer x $\ge$ 0
\\garante sumaux(n,t) = $$\sum_{i=1}^{n} i^{2} + t$$ 
\\invariante $\forall$k $\in$ N. sumaux(w0,u0) = $$\sum_{i=1}^{wk} i^{2} + uk$$ 
\\sumaux01: sumaux(0,a) = a
\\sumaux02: sumaux(n,a) = sumaux(n - 1,n$^{2}$ + a), se n $\textgreater$ 0

\subsubsection{Exemplo de execução}
somatorio(4) 
\\= sumaux(4,0)
\\= sumaux(3,4 * 4 + 0)
\\= sumaux(2,3 * 3 + 4 * 4 + 0)
\\= sumaux(1,2 * 2 + 3 * 3 + 4 * 4 + 0)
\\= sumaux(0,1 * 1 + 2 * 2 + 3 * 3 + 4 * 4 + 0)
\\= 1 * 1 + 2 * 2 + 3 * 3 + 4 * 4 + 0
\\= 30

\subsubsection{Teorema th01: $$\sum_{i=1}^{n} i^{2} = (n/6) * (n + 1) * (2 * n + 1)$$ }
P $\triangleq$ $\forall$n $\in$ N. $$\sum_{i=1}^{n} i^{2} = (n/6) * (n + 1) * (2 * n + 1)$$
\\P(n) $\triangleq$ $$\sum_{i=1}^{n} i^{2} = (n/6) * (n + 1) * (2 * n + 1)$$
\\
\textbf{Base:}
\\P(0) $\triangleq$ $$\sum_{i=1}^{0} i^{2} = (0/6) * (0 + 1) * (2 * 0 + 1)$$
\\
\\Mostrar $$\sum_{i=1}^{0} i^{2} = (0/6) * (0 + 1) * (2 * 0 + 1)$$
\\\textit{Prova:}
\\0 (por aritmética)
\\ (n/6) * (n + 1) * (2 * n + 1) (por aritmética)
\\
\\
\textbf{Indução:}
Seja n0 uma variável arbitrária
\\Assumir como hipótese P(n0) $\triangleq$ $$\sum_{i=1}^{n0} i^{2} = (n0/6) * (n0 + 1) * (2 * n0 + 1)$$
\\Demonstrar P(n0 + 1) $\triangleq$ $$\sum_{i=1}^{n0 + 1} i^{2} = (n0 + 1/6) * (n0 + 1 + 1) * (2 * (n0 + 1) + 1)$$
\\Mostrar $$\sum_{i=1}^{n0 + 1} i^{2} = (n0 + 1/6) * (n0 + 1 + 1) * (2 * (n0 + 1) + 1)$$
\\\textit{Prova:}
\\$$\sum_{i=1}^{n0 + 1} i^{2} + (n0 + 1)^{2}$$ (por aritmética)
\\(n0/6) * (n0 + 1) * (2 * n0 + 1) + (n0 + 1)$^{2}$ (por hipótese)
\\((n0 + 1)/6) * (n0 + 2) * (2 * (n0 + 1) + 1) (por aritmética)


\subsubsection{Teorema th02: sumaux(n, t) = $$\sum_{i=1}^{n} i^{2} + t$$}
P $\triangleq$ $\forall$n,t $\in$ N. sumaux(n,t) = $$\sum_{i=1}^{n} i^{2} + t$$
\\P(n) $\triangleq$ $\forall$t $\in$ N. sumaux(n,t) = $$\sum_{i=1}^{n} i^{2} + t$$
\\
\\\textbf{Base:}
\\P(0) $\triangleq$ $\forall$t $\in$ N. sumaux(0,t) = $$\sum_{i=1}^{0} i^{2} + t$$
\\
\\\textit{Prova:}
\\Seja t0 uma variável arbitrária
\\Mostrar sumaux(0,t0) = $$\sum_{i=1}^{0} i^{2} + t0$$
\\t0 (por sumaux01)
\\i=$$\sum_{i=1}^{0} i^{2} + t0$$ (por aritmética)
\\
\\\textbf{Indução:}
\\Seja n0 uma variável arbitrária
\\Assumir como hipótese \\P(n0) $\triangleq$ $\forall$t $\in$ N. sumaux(n0,t) = $$\sum_{i=1}^{n0} i^{2} + t$$
\\Demonstrar \\P(n0 + 1) $\triangleq$ $\forall$t $\in$ N. sumaux(n0 + 1,t) = $$\sum_{i=1}^{n0 + 1} i^{2} + t$$
\\
\\\textit{Prova:}
\\Seja t0 uma variável arbitrária 
\\Mostrar sumaux(n0 + 1,t0) = $$\sum_{i=1}^{n0 + 1} i^{2} + t0$$
\\sumaux(n0,(n0 + 1)$^{2}$ + t0) (por sumaux02)
\\$$\sum_{i=1}^{n0 + 1} (n0 + 1)^{2} + t0$$ (por hipótese)
\\$$\sum_{i=1}^{n0 + 1} i^{2} + t0$$ (por aritmética)

\subsubsection{Teorema th03: somatorio(n) = $$\sum_{i=1}^{n} i^{2}$$}
P $\triangleq$ $\forall$n $\in$ N. somatorio(n) = $$\sum_{i=1}^{n} i^{2}$$
\\P(n) $\triangleq$ $\forall$n $\in$ N. somatorio(n) = $$\sum_{i=1}^{n} i^{2}$$
\\
\\\textbf{Base:}
\\P(0) $\triangleq$ $\forall$n $\in$ N. somatorio(0) = $$\sum_{i=1}^{0} i^{2}$$
\\
\\\textit{Prova:}
\\Mostrar somatorio(0) = $$\sum_{i=1}^{0} i^{2}$$
\\sumaux(0,0) (por definição)
\\0 (por definição)
\\$$\sum_{i=1}^{0} i^{2}$$ (por aritmética)
\\
\\\textbf{Indução:}
\\Seja n0 uma variável arbitrária
\\Assumir como hipótese P(n0) $\triangleq$ somatorio(n0) = $$\sum_{i=1}^{n0} i^{2}$$
\\Demonstrar P(n0 + 1) $\triangleq$ somatorio(n0 + 1) = $$\sum_{i=1}^{n0 + 1} i^{2}$$
\\
\\\textit{Prova:}
\\Mostrar somatorio(n0 + 1) = $$\sum_{i=1}^{n0 + 1} i^{2}$$
\\sumaux(n0 + 1,0) (por definição)
\\sumaux(n0,(n0 + 1)$^{2}$ + 0) (por definição)
\\$$\sum_{i=1}^{n0} (n0 + 1)^{2} + 0$$ (por definição da invariante sumaux)
\\$$\sum_{i=1}^{n0 + 1} i^{2}$$(por aritmética)
\\
\subsubsection{Teorema: sum(n) = (n/6) * (n + 1) * (2 * n + 1)}
P $\triangleq$ $\forall$n $\in$ N. sum(n) = (n/6) * (n + 1) * (2 * n + 1)
\\P(n) $\triangleq$ $\forall$n $\in$ N sum(n) = (n/6) * (n + 1) * (2 * n + 1)
\\
\\\textbf{Base:}
\\P(0) $\triangleq$ sum(0) = (0/6) * (0 + 1) * (2 * 0 + 1) 
\\
\\\textit{Prova:}
\\Mostrar sum(0) = (0/6) * (0 + 1) * (2 * 0 + 1) 
\\sumaux(0,0) (por definição)
\\0 (por definição)
\\(0/6) * (0 + 1) * (2 * 0 + 1) (por aritmética)
\\
\\\textbf{Indução:}
\\Seja n0 uma variável arbitrária
\\Assumir como hipótese P(n0) $\triangleq$ sum(n0) = (n0/6) * (n0 + 1) * (2 * n0 + 1)
\\Demonstrar P(n0 + 1) $\triangleq$ sum(n0 + 1) = ((n0 + 1)/6) * (n0 + 2) * (2 * (n0 + 1) + 1)
\\
\\\textit{Prova:}
\\Mostrar sum(n0 + 1) = ((n0 + 1)/6) * (n0 + 2) * (2 * (n0 + 1) + 1)
\\sumaux(n0 + 1,0) (por definição)
\\sumaux(n0,(n0 + 1)$^{2}$ + 0) (por definição)
\\$$\sum_{i=1}^{n0} i^{2} + (n0 + 1)^{2} + 0$$ (por definição da invariante sumaux)
\\sum(n0) + (n0 + 1)$^{2}$ (por definição sum(n) = $$\sum_{i=1}^{n0} i^{2}$$)
\\(n0/6) * (n0 + 1) * (2 * n0 + 1) + (n0 + 1)$^{2}$ (por hipótese)
\\((n0 + 1)/6) * (n0 + 2) * (2 * (n0 + 1) + 1) (por aritmética)

\section{Terceiro problema}
\subsection{Especificação recursiva na cauda}
\textbf{tail\_len\_c}: List $\rightarrow$ N
\\requer L
\\garante tail\_len\_c L = length L
\\\textit{tail\_len\_c01}: tail\_len\_c [] = 0
\\\textit{tail\_len\_c02}: tail\_len\_c T = tail\_len\_c\_aux T 0
\\\textit{onde}
\\
\textbf{tail\_len\_c\_aux}: List x N $\rightarrow$ N
\\requer L
\\\textit{tail\_len\_c\_aux01}: tail\_len\_c\_aux [] a = a
\\\textit{tail\_len\_c\_aux02}: tail\_len\_c\_aux (h\#T) a = tail\_len\_c\_aux T (Suc(a))

\subsubsection{Exemplo de execução}
tail\_len\_c [4, 7]
\\= tail\_len\_c\_aux [4, 7] 0
\\= tail\_len\_c\_aux [7] 1
\\= tail\_len\_c\_aux [] 2
\\= 2

\subsection{Especificação não recursiva na cauda}
\textbf{tail\_len\_nc}: List $\rightarrow$ N
\\requer L
\\garante tail\_len\_nc L = length L
\textit{tail\_len\_nc01}: tail\_len\_nc [] = 0
\textit{tail\_len\_nc02}: tail\_len\_nc (h\#T) = 1 + tail\_len\_nc T

\subsubsection{Exemplo de execução}
tail\_len\_nc [4, 5]
\\= tail\_len\_nc (1 + tail\_len\_nc [5])
\\= 1 + (tail\_len\_nc (1 + tail\_len\_nc []))
\\= (1 + 1 + 0)
\\= (2 + 0)
\\= 2

\subsection{Teorema th01: tail\_len\_nc l = length l}
P $\triangleq$ tail\_len\_nc l = length l
\\P(l) $\triangleq$ tail\_len\_nc l = length l
\\
\\\subsubsection{Base:} P([]) $\triangleq$ tail\_len\_nc
\\
\\Mostrar tail\_len\_nc [] = length []
\\
\\\textit{Prova:}
\\tail\_len\_nc [] = 0 (por tail\_len\_nc01)
\\length [] (por simplificação)
\\
\\\subsubsection{Indução:}
Seja l0 uma variável arbitrária
\\Seja \textit{e} uma variável arbitrária
\\Assumir como hipótese P(l0) $\triangleq$ tail\_len\_nc l0 = length l0
\\Demonstrar P(e\#l0) $\triangleq$ tail\_len\_nc (e\#l0) = length (e\#l0)
\\
\\\textit{Prova:}
\\tail\_len\_nc l0 + 1 (por tail\_len\_nc02)
\\length l0 + 1 (por hipótese)
\\length (e\#l0) (por simplificação)

\subsection{Teorema th02: $\forall$a. tail\_len\_c\_aux l a = length l + a}
P $\triangleq$ $\forall$a $\in$ N. tail\_len\_c\_aux l a = length l + a
\\P(l) $\triangleq$ $\forall$a $\in$ N. tail\_len\_c\_aux l a = length l + a
\\
\\\subsubsection{Base:} P([]) $\triangleq$ $\forall$a $\in$ N. tail\_len\_c [] a = length [] + a
\\Seja a0 uma variável arbitrária
\\Mostrar tail\_len\_c\_aux [] a0 = length [] + a0
\\
\\\textit{Prova:}
\\a0 (por tail\_len\_c\_aux01)
\\length [] + a0 (por simplificação)
\\
\\\subsubsection{Indução:}
Seja l0 uma variável arbitrária
\\Seja \textit{elem} uma variável arbitrária
\\Assumir como hipótese P(l0) $\triangleq$ $\forall$a. tail\_len\_c\_aux l0 a = length l0 + a
\\Demonstrar P(elem\#l0) $\triangleq$ $\forall$a. tail\_len\_c\_aux (elem\#l0) a = length (elem\#l0) + a
\\
\\Seja a0 uma variável arbitrária
\\Mostrar P(elem\#l0) $\triangleq$ tail\_len\_c\_aux (elem\#l0) a0 = length (elem\#l0) + a0
\\
\\\textit{Prova:}
\\tail\_len\_c\_aux l0 (Suc a0) (por tail\_len\_c\_aux02)
\\length l0 + (Suc a0) (por hipótese)
\\length (elem\#l0) + a0 (por simplificação)

\subsection{Teorema th03: tail\_len\_nc T = tail\_len\_c T}
P $\triangleq$ tail\_len\_nc T = tail\_len\_c T
\\P(l) $\triangleq$ tail\_len\_nc l = tail\_len\_c l
\\
\\\subsubsection{Base:} P([]) $\triangleq$ tail\_len\_nc [] = tail\_len\_c []
\\
\\\textit{Prova:}
\\tail\_len\_nc [] = 0 (por tail\_len\_nc01)
\\tail\_len\_c [] (por tail\_len\_c01)
\\
\\\subsubsection{Indução:}
Seja T0 uma variável arbitrária
\\Seja a0 uma variável arbitrária
\\Assumir como hipótese P(T0) $\triangleq$ tail\_len\_nc T0 = tail\_len\_c T0
\\Demonstrar P(a0\#T0) $\triangleq$ tail\_len\_nc (a0\#T0) = tail\_len\_c (a0\#T0)
\\
\\\textit{Prova:}
\\length (a0\#T0) (por th01)
\\= tail\_len\_c\_aux (a0\#T0) 0 (por th02)
\\= tail\_len\_c (a0\#T0)" (por tail\_len\_c02)

\end{document}
